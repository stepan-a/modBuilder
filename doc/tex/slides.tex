\documentclass[10pt,slidestop]{beamer}
\synctex=1

\usepackage{etex}
\usepackage{fourier-orns}
\usepackage{ccicons}
\usepackage{amssymb}
\usepackage{amstext}
\usepackage{amsbsy}
\usepackage{amsopn}
\usepackage{amscd}
\usepackage{amsxtra}
\usepackage{amsthm}
\usepackage{float}
\usepackage{color}
\usepackage{mathrsfs}
\usepackage{bm}
\usepackage[nice]{nicefrac}
\usepackage{setspace}
\usepackage{hyperref}
\usepackage{ragged2e}
\usepackage{listings}
\usepackage{fontawesome}
\usepackage{environ}
\usepackage{manfnt}

%\usepackage[utf8x]{inputenc}
\usepackage[T1]{fontenc}
\usepackage{array}
\usepackage[super]{nth}
\usepackage{algorithm}
\usepackage{algpseudocode}
\newcolumntype{x}[1]{>{\centering\arraybackslash\hspace{0pt}}m{#1}}
\setlength{\parindent}{0pt}
\pdfcompresslevel=9

\usepackage{verbatimbox}
\usepackage{pgfplots}
\usepackage{matlab-prettifier}
\usepackage{listings}
\usepackage{xcolor}

\lstdefinestyle{MatlabConsole}{
  basicstyle=\ttfamily\small,
  backgroundcolor=\color{black!5},
  frame=single,
  breaklines=true,
  numbers=none,
  showstringspaces=false,
}

\lstdefinelanguage{Dynare}
{
  % list of keywords
  morekeywords={
    var_model,
    trend_component_model,
    var_expectation_model,
    var_expectation,
    pac_model,
    pac_expectation,
    model,
    diff,
    end,
    varexo,
    var,
    parameters,
    steady_state_model,
    steady,
    shocks,
    perfect_foresight_setup,
    perfect_foresight_solver,
    initval,
    endval,
  },
  sensitive=false, % keywords are not case-sensitive
  morecomment=[l]{//}, % l is for line comment
  morecomment=[s]{/*}{*/}, % s is for start and end delimiter
  morestring=[b]'          % defines that strings are enclosed in double quotes
}

\definecolor{eclipseBlue}{RGB}{42,0.0,255}
\definecolor{eclipseGreen}{RGB}{63,127,95}
\definecolor{eclipsePurple}{RGB}{127,0,85}
\definecolor{eclipseEmph}{RGB}{0,0,0}

% Set Language
\lstset{
  language={Dynare},
  inputencoding = utf8,
  basicstyle=\footnotesize\ttfamily, % Global Code Style
  captionpos=b, % Position of the Caption (t for top, b for bottom)
  extendedchars=true, % Allows 256 instead of 128 ASCII characters
  tabsize=2, % number of spaces indented when discovering a tab
  columns=fixed, % make all characters equal width
  keepspaces=true, % does not ignore spaces to fit width, convert tabs to spaces
  showstringspaces=false, % lets spaces in strings appear as real spaces
  breaklines=true, % wrap lines if they don't fit
  frame=trbl, % draw a frame at the top, right, left and bottom of the listing
  frameround=tttt, % make the frame round at all four corners
  framesep=4pt, % quarter circle size of the round corners
  numbers=left, % show line numbers at the left
  numberstyle=\tiny\ttfamily, % style of the line numbers
  commentstyle=\color{eclipseGreen}, % style of comments
  keywordstyle=\color{eclipsePurple}\bfseries, %, % style of keywords
  stringstyle=\color{eclipseBlue}, % style of strings
  emph={model_name, periods, eqtags, targets, expression, auxiliary_model_name, horizon,discount},emphstyle=\color{eclipseEmph}\bfseries,
  literate={⟂}{{\ensuremath{\perp}}}{1},
}



\newcommand{\trace}{\mathrm{tr}}
\newcommand{\vect}{\mathrm{vec}}
\newcommand{\tracarg}[1]{\mathrm{tr}\left\{#1\right\}}
\newcommand{\vectarg}[1]{\mathrm{vec}\left(#1\right)}
\newcommand{\vecth}[1]{\mathrm{vech}\left(#1\right)}
\newcommand{\iid}[2]{\mathrm{iid}\left(#1,#2\right)}
\newcommand{\normal}[2]{\mathcal N\left(#1,#2\right)}
\newcommand{\dynare}{\href{http://www.dynare.org}{\color{blue}Dynare}}
\newcommand{\AllSample}{ \mathcal Y_T }
\newcommand{\sample}{\mathcal Y_T}
\newcommand{\samplet}[1]{\mathcal Y_{#1}}
\newcommand{\slidetitle}[1]{\fancyhead[L]{\textsc{#1}}}

\definecolor{gray}{gray}{0.4}

\theoremstyle{plain}

\makeatletter
\@ifclassloaded{beamer}{
\setbeamertemplate{footline}{
  {\hfill\vspace*{1pt}\href{https://creativecommons.org/publicdomain/zero/1.0/}{\cczero}\hspace{.1cm}
    \href{https://mnemosyne.ithaca.fr/stephane/ispra-2025}{\faCodeFork}\enspace--\enspace\today\enspace
  }}
\setbeamertemplate{navigation symbols}{}
\setbeamertemplate{blocks}[rounded][shadow=true]
\setbeamertemplate{caption}[numbered]}
\makeatother

% From https://tex.stackexchange.com/questions/312118/a-note-with-the-dangerous-bend-symbol-how-to-vertically-center-text-and-symbol
\newcommand{\dnote}[1]{%
    \noindent % I guess this is intended...
    \begin{tabular}{@{}m{0.13\textwidth}@{}m{0.87\textwidth}@{}}%
        \huge\textdbend &#1%
    \end{tabular}%
    \par % ... and this too.
}

\NewEnviron{notes}{\justifying\footnotesize\begin{spacing}{1.0}\BODY\vfill\pagebreak\end{spacing}}

\begin{document}

\title{modBuilder}
\author[]{\texttt{stephane.adjemian@univ-lemans.fr}}
\date{October,  2025}

\begin{frame}
  \titlepage{}
\end{frame}

\begin{frame}[fragile]
  \frametitle{Introduction}

  \bigskip

  \begin{itemize}

  \item \verb+modBuilder+ is a MATLAB class aimed at simplifying the interactive and
    programmatic creation of Dynare \verb+.mod+ files.\newline

  \item Build incrementally a model directly from the MATLAB.\newline

  \item Facilitates consistency checks, error detection, and modular
    model development.\newline

  \item Also provides tools for computing the steady state.\newline

  \item Export the model to a \verb+.mod+ file ready to run.\newline

  \item Code available in a public \href{https://git.dynare.org/stepan-a/modbuilder}{Git repository} (not part of Dynare).

  \end{itemize}

\end{frame}


\begin{frame}[fragile]
  \frametitle{Instantiate an empty model}

\begin{lstlisting}[style=MatlabConsole]
>> model = modBuilder()

model =

  modBuilder with properties:

       params: {0x4 cell}
       varexo: {0x4 cell}
          var: {0x4 cell}
      symbols: {1x0 cell}
    equations: {0x2 cell}
            T: [1x1 struct]
         date: 15-Sep-2025 15:36:06
   \end{lstlisting}

\end{frame}


\begin{frame}[fragile]
  \frametitle{Instantiate a model from a \texttt{.mod} file -- I --}

  \lstinputlisting[firstline=1,lastline=19,frame=none,basicstyle=\footnotesize,language=Dynare]{../models/rbc12.mod}

\end{frame}

\begin{frame}[fragile]
  \frametitle{Instantiate a model from a \texttt{.mod} file -- II --}

  \lstinputlisting[firstline=21,lastline=41,frame=none,basicstyle=\footnotesize,language=Dynare]{../models/rbc12.mod}

\end{frame}


\begin{frame}[fragile]
  \frametitle{Instantiate a model from a \texttt{.mod} file -- III --}

  \begin{lstlisting}[style=MatlabConsole]

    >> dynare rbc12 json=compute
    ...
    >> model = modBuilder(M_, oo_, 'rbc12/model/json/modfile-original.json')

model =

  modBuilder with properties:

       params: {8x4 cell}
       varexo: {2x4 cell}
          var: {6x4 cell}
      symbols: {1x0 cell}
    equations: {6x2 cell}
            T: [1x1 struct]
         date: 15-Sep-2025 16:38:38
   \end{lstlisting}

\end{frame}


\begin{frame}[c,fragile]
  \frametitle{Instantiate a model}

  \begin{itemize}

  \item It is also possible to create a new model from a previously defined modBuilder object that has been saved to disk.\newline

  \item Once a \verb+modBuilder+ object is instantiated, the model can be expanded or transformed by utilizing the methods provided by the class.\newline

  \item In the end the model can be saved on disk for later usage (\verb+save+ method) or exported to a \verb+.mod+ file (\verb+write+ method).

  \end{itemize}

\end{frame}


\begin{frame}[c,fragile]
  \frametitle{Add an equation}

  \begin{lstlisting}[style=MatlabConsole]

    >> model.add('y', 'y = rho*y(-1)-gamma*z+e');
   \end{lstlisting}

   \bigskip

  \begin{itemize}

  \item Each equation must be associated to an endogenous variable (\verb+y+) that will be used to index the model object.\newline

  \item[$\Rightarrow$] Each new equation adds a new endogenous variable to the model.\newline

  \item All the other symbols (\verb+rho+, \verb+z+, and \verb+e+), unless previously defined, are classified as unresolved.\newline

  \item These symbols need to be typed before completing the model. This can be done by adding an equation for an endogenous variable (\verb+z+) or explictly declaring a symbol as a parameter (\verb+rho+) or an exogenous variable (\verb+e+).

  \end{itemize}

\end{frame}


\begin{frame}[c,fragile]
  \frametitle{Type symbols}

  \begin{itemize}

  \item \verb+model.symbols+ is the list of unclassified symbols.\newline

  \item This property must be empty to finalize the model.\newline

  \end{itemize}

  \begin{lstlisting}[style=MatlabConsole]

    >> model.parameter('rho', .9, 'texname', '\rho');
    >> model.exogenous('epsilon', .0);
   \end{lstlisting}

   \medskip

  \begin{itemize}

  \item Calibration occurs upon declaring a parameter but can be modified subsequently.\newline

  \item The value assigned to an exogenous variable determines its long-run level.

  \end{itemize}

\end{frame}


\begin{frame}[c,fragile]
  \frametitle{Converting symbols}

  \begin{itemize}

  \item Methods \verb+parameter+ and \verb+exogenous+ can also be used to convert a symbol.\newline

  \item These methods cannot be applied to an endogenous variable.\newline

  \item To exogenize an endogenous variable one can:
    \begin{itemize}
    \item[--] Remove an equation (methods \verb+rm+ or \verb+remove+)
    \item[--] Flip an exogenous variable and and endogenous variable (\verb+flip+ method).
    \end{itemize}

  \end{itemize}

    \bigskip

    \dnote{The method \texttt{endogenous} does not change the type of a variable, but only serves to provide the steady state level of an endogenous variable.}

    \begin{lstlisting}[style=MatlabConsole]

    >> model.endogenous('y', .0);
  \end{lstlisting}

\end{frame}


\begin{frame}[c,fragile]
  \frametitle{Change an equation}

  \begin{itemize}
  \item \verb+change+ method: change a whole equation.
  \item \verb+subs+ method: substitute an expression in a set of equations.
  \item \verb+substitute+ method: same as \verb+subs+ but with a regular expression.
  \item \verb+rename+ method: rename a symbol in all equations.
  \end{itemize}

    \medskip

    \dnote{It is recommended to use regular expressions with the \texttt{subs} method; consult the Mathworks documentation for more information: \href{https://fr.mathworks.com/help/matlab/ref/regexprep.html}{regexprep}.}

    \bigskip

    \dnote{If new symbols are introduced with the changes, they must be typed after.}

    \bigskip

    \begin{lstlisting}[style=MatlabConsole]

      >> model.change('y', 'y=rho*y(-1)+phi*y(-2)+e');
      >> model.subs('y(-2)', 'y(-3)', 'y');
  \end{lstlisting}

\end{frame}


\begin{frame}[c,fragile]
  \frametitle{Search for symbols within a model}

  \begin{itemize}

  \item The \verb+lookfor+ method searches for any symbol, such as a parameter or variable, within a model and displays the equations in which the symbol appears.\newline

  \end{itemize}

    \medskip

    \begin{lstlisting}[style=MatlabConsole]

      >> model.lookfor('k')

      Endogenous variable k appears in 3 equations:

      1/beta = ((exp(b)*c)/(exp(b(+1))*c(+1)))*(exp(b(+1))*alpha*y(+1)/k+(1-delta))

      y = exp(a)*(k(-1)^alpha)*(h^(1-alpha))

      k = exp(b)*(y-c)+(1-delta)*k(-1)
  \end{lstlisting}

\end{frame}


\begin{frame}[c,fragile]
  \frametitle{Use matlab to write an equation -- I --}

  \begin{itemize}

  \item The features of the MATLAB language, such as loops and conditional statements, can be employed to programmatically generate equations.\newline

  \item Suppose we want to add the following equation:
    \[
      y_t = \rho_1 y_{t-1} + \rho_2 y_{t-2} + \dots + \rho_{12} y_{t-12} + \varepsilon_t
    \]

    \medskip

  \item Define a general MATLAB routine generating AR equations:\newline

    \lstinputlisting[
frame=none,
numbers=left,
basicstyle=\small,
style=Matlab-editor
]{../models/ar.m}

  \end{itemize}

\end{frame}


\begin{frame}[c,fragile]
  \frametitle{Use matlab to write an equation -- II --}

  \lstinputlisting[
  firstline=1,
  lastline =8,
  basicstyle=\small,
  frame=none,
  numbers=left,
  style=Matlab-editor
  ]{../models/ar12.m}

  \bigskip

  \begin{lstlisting}[style=MatlabConsole]

      >> m.y.equations{2}

      ans =

      'y = rho1*y(-1) + rho2*y(-2) + rho3*y(-3) + rho4*y(-4) + rho5*y(-5) + rho6*y(-6) + rho7*y(-7) + rho8*y(-8) + rho9*y(-9) + rho10*y(-10) + rho11*y(-11) + rho12*y(-12) + e'
  \end{lstlisting}

\end{frame}


\begin{frame}[c,fragile]
  \frametitle{Use matlab to write equations -- III --}

  \lstinputlisting[
  firstline=1,
  lastline =10,
  basicstyle=\tiny,
  frame=none,
  numbers=left,
  style=Matlab-editor
  ]{../models/ar12panel.m}

\end{frame}


\begin{frame}[c,fragile]
  \frametitle{Steady state -- 1 --}

  \begin{itemize}

  \item The \verb+evaluate+ method evaluates a static equation and
    returns a structure with the value of the left hand side (field
    \verb+lhs+), the value of the right hand side (field \verb+rhs+)
    and the residual of the static equation defined as
    \verb+resid=lhs-rhs+.\newline

    \medskip

  \item The \verb+solve+ method solves the static version of an
    equation for a symbol. This symbol can be an endogenous variable,
    an exogenous variable or a parameter.\newline

  \end{itemize}

\end{frame}


\begin{frame}[c,fragile]
  \frametitle{Steady state -- 2 --}

  \scalebox{.6}{
  \lstinputlisting[
  firstline=1,
  lastline =24,
  basicstyle=\tiny,
  frame=none,
  numbers=left,
  style=Matlab-editor
  ]{../models/rbc13.m}}

\end{frame}

\begin{frame}[c,fragile]
  \frametitle{Steady state -- 3 --}

  \scalebox{1}{
  \lstinputlisting[
  firstline=28,
  lastline =37,
  basicstyle=\tiny,
  frame=none,
  numbers=left,
  style=Matlab-editor
  ]{../models/rbc13.m}}

\end{frame}


\end{document}


% Local Variables:
% ispell-check-comments: exclusive
% ispell-local-dictionary: "american-insane"
% TeX-master: t
% End:
